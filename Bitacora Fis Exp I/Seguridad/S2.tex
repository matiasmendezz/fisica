\begin{titlepage}
   \begin{center}
       \vspace*{9cm}
       \LARGE
       \textbf{Seguridad 2: Fuego}

      
   
            
       \vspace{0.8cm}
     
      
   \end{center}
\end{titlepage}
\section{Fuego }
\subsection{Conceptos}
\begin{enumerate}
  \item Según la Academia Nacional de Bomberos de Chile(ANB), el fuego es una reacción química continua, con generación de luz y calor, en la que se combinan elementos combustibles (agentes reductores) y el oxígeno (agente oxidante). Para esto se requiere la presencia de una fuente de calor y cantidades adecuadas de combustibles y oxígeno. 
  \item %Pregunta 2
  \begin{enumerate}
      \item Fuego y llama no son sinónimos, técnicamente la llama es la parte visible del fuego, es decir son los gases incandescentes que se desprenden de la combustión. Esta se da cuando la temperatura generada es suficientemente alta como para ionizar los vapores a su alrededor. 
      \item Puede haber fuego sin llama, este se llama fuego incandescente. Esto ocurre cuando el combustible, el oxígeno y el calor están en cantidades apropiadas. El oxígeno que participa en la combustión es relativamente pequeño, y está en contacto con la superficie del combustible, el cual debe estar en estado sólido.  Ejemplo: la brasa, que corresponde al resultado de los momentos finales de combustión de la madera. Si soplamos la brasa aumenta la cantidad de oxígeno, con lo que se puede producir llamas.  
  \end{enumerate}
  
  \item  %Pregunta 3
  \begin{enumerate}
      \item \textbf{Triángulo de fuego}: el fuego incandescente necesita tres elementos indispensables para su combustión. Cada lado del triángulo simboliza uno de estos elementos: \textbf{combustible}, \textbf{comburente} y \textbf{calor}.
      \\El \textbf{combustible} es cualquier material que pueda actuar como agente reductor(que cede electrones ), es decir toda sustancia susceptible a arder. 
      \\El \textbf{comburente }es el agente oxidante(que recibe electrones), es cualquier sustancia que en ciertas condiciones de temperatura y presión puede combinarse con un combustible, provocando así una combustión. Actúa oxidando el combustible, y por lo tanto siendo reducido por este último. En general se usa el oxígeno como comburente. 
      \\El \textbf{calor} es una forma de energía presente en todos los objetos materiales, aunque en mayor o menor cantidad. La cantidad de calor se expresa mediante la temperatura. El calor siempre se transfiere de los cuerpos con mayor temperatura a los que están a una temperatura menor. 
      \\Al eliminar uno de los componentes del triángulo, el fuego incandescente se extingue. En cambio, si se aumenta la cantidad de oxígeno, se producen llamas con lo que ya no se puede explicar con el triángulo de fuego, así que se usa el tetraedro de fuego. 
      \item \textbf{Tetraedro de fuego}: es una pirámide triangular en que cada una de sus cuatros superficies se identifica cada unos de los componentes para que el fuego continúe y se produzcan llamas, además de los tres componentes anteriores (triángulo de fuego) es necesaria una reacción en cadena. Esta última es la que explica la producción y mantenimiento de las llamas.  
    \\Para que la combustión con llama se sostenga, el fuego original tiene que generar suficiente calor como para garantizar la existencia de vapores. Los nuevos vapores, al mezclarse con el oxígeno, generan una llama mayor, con más calor, lo que a su vez generan más vapores, y así sucesivamente. En este proceso, una parte del calor producido se transmite al medio ambiente, pero otra vuelve al mismo proceso (lo “retro-alimenta”), produciendo la Reacción en Cadena. Puede ocurrir cuando los otros tres elementos se encuentran en la proporción adecuada.	
  \end{enumerate}
  
  \item Tanto el triángulo de fuego como el tetraedro nos dicen que el fuego se puede generar o mantener al estar todos estos componentes unidos, por lo que al quitar uno de los componentes se puede detener o evitar el fuego.
  
  \item Según norma NCH 934, los tipos de fuego son: %pregunta 5
  \begin{enumerate}
      \item Clase A: Fuegos de materiales combustibles sólidos comunes: madera, papel, telas, diversos plásticos, etc. En general son todos aquellos materiales que al arder dejan brasas o cenizas. Se extinguen mediante el enfriamiento, es decir, eliminando la componente temperatura. Para esto, el agua es la sustancia extintora ideal.
      \item Clase B: Fuego de líquidos combustibles o inflamables: aceites, alcohol, grasas, alquitrán, pinturas, gases inflamables. Se apagan eliminando el oxígeno o interrumpiendo la reacción en cadena.
      \item Clase C: Fuego en equipos eléctricos o materiales energizados. Involucran equipos eléctricos o cualquier otro combustible (fuego clase A,B o D) energizado. El agente extintor no debe ser conductor de electricidad como el agua o la espuma que contiene agua. Una vez desenergizado y verificada la ausencia de electricidad, se puede extinguir con agua. Si no es posible desenergizar, sólo usar agentes extintores no conductores de la electricidad como el Polvo Químico Seco o el Dióxido de Carbono.
      \item Clase D: Fuego en metales combustibles:  magnesio, titanio, litio, zirconio,  sodio, potasio, etc; que al arder alcanzan temperaturas muy elevadas(2700°C a 3300°C).  Se apagan con un matafuego cargado con agente extintor de polvo.
  \end{enumerate}
  \item % pregunta 6
  \begin{enumerate}
      \item La Temperatura de Gasificación es la temperatura mínima a la cual un combustible sólido o líquido desprende vapores en cantidad suficiente para formar una mezcla inflamable con el aire ambiente y permitir la combustión. Los gases no necesitan temperatura de gasificación porque ya están en dicho estado.
      \item Si la temperatura del gas aumenta, por estar en contacto con un material que tenga una determinada temperatura, llegará a un punto en que comenzará a arder con una combustión sostenida.  La temperatura de ignición es la temperatura mínima a la cual los vapores del combustible comienzan a arder.
  \end{enumerate}
  \item %pregunta 7
  \begin{enumerate}
      \item 
      \begin{itemize}
      \item Líquido inflamable: son aquellos cuya Temperatura de Gasificación es inferior a 37 grados Cº. Por ejemplo: la gasolina y el alcohol etílico.
      \item Líquido combustible: son aquellos cuya Temperatura de Gasificación es igual o superior a 37 grados Cº. Por ejemplo: el queroseno y el petróleo.
      \end{itemize}
      \item Ejemplos de combustibles no inflamables son la madera o  parafina a temperatura ambiente.
  \end{enumerate}
  \item Es el rango en cual el porcentaje del vapor del combustible en mezcla con el aire permite la combustión. Los porcentajes maximo y minimos de vapor combustible que permiten la combustión se denominan límites superior e inferior respectivamente. fuera de estos límites no hay combustión.
  \item %pregunta 9
  \begin{enumerate}
      \item La explosión volumétrica es causada por la ebullición violenta de agua, el vapor de agua ocupa un volumen 1800 veces mayor que el agua por lo que la burbuja de vapor desplaza la materia en combustión, produciendo así una explosión. Esto ocurre, cuando se trata de apagar con agua un incendio que no se debe apagar de ese modo.
      \item Un ejemplo doméstico es al verter agua en aceite hirviendo.
  \end{enumerate}
  \end{enumerate}
\subsection{Peligros}

\begin{enumerate}[resume]
  \item  Si los materiales están compuestos por celulosas o fibras artificiales, el humo será de color gris y negro en los plásticos , petróleo, materiales acrílicos. Y puede indicar que el fuego arde en presencia de poco oxígeno. La oscuridad se puede producir por la generación de humos y también por un eventual corte de energía eléctrica.\\ \\
  La oscuridad produce desorientación, lo que dificulta una evacuación rápida y segura. De igual forma dificulta el trabajo de los equipos de extinción y la extracción de los posibles heridos en el incendio. 
  
  \item El humo es el resultado de la combustión incompleta. Las principales componentes del humo son: vapor de agua, hidrocarburos, dióxido de carbono, monóxido de carbono, hollín y otros gases (cianuro de hidrógeno, cloruro de hidrógeno, fosgeno, fosfina, dióxido de azufre, amoníaco, cloro, etc ) que dependen de la composición del material que se quema. \\ \\
  Los peligros que provocan el humo son la asfixia, pérdida de conciencia, dificultad de la visión lo que también provocaría desorientación, en general estos gases están a altas temperaturas por lo que pueden provocar quemaduras.
  \item Lo ideal y crucial es una reacción rápida de la persona para la oportuna extinción del incendio. En caso de que la persona sea incapaz de apagar el incendio por sí sola, debe evacuar el lugar inmediatamente de forma rápida y segura, agachado con un paño humedecido tapando nariz y boca. 
  \item 
  El calor es otro de los peligros que afectan a las personas en un incendio.En un incendio las temperaturas pueden alcanzar los miles de grados por lo que el riesgo de quemaduras graves es muy alto. Hay que tener especial cuidado al abrir puertas ya que en general no se sabe con certeza a qué lugares ha llegado el fuego. Si la visión se ve afectada lo mejor es estar lo más cercano al piso ya que los gases a altas temperaturas permanecen en la altura.
\end{enumerate}

\subsection{Extintores}
Los extintores son un aparato portátil para apagar fuegos o incendios de pequeña magnitud que consiste en una especie de botella grande en cuyo interior hay una sustancia líquida, espumosa o en forma de polvo, para apagar el fuego se arroja un chorro de esta sustancia.
\begin{enumerate}[resume]
  
  \item  Extintor en base a agua: consiste en agua mezclada con algún aditivo químico que sofoque el fuego. Su función principal es absorber el calor. Es útil contra incendios clase A.
  \item Extintor con dióxido de carbono: contribuye a disminuir el oxígeno del aire por lo que este extintor elimina el comburente, al salir el gas alcanza temperaturas muy bajas por lo que también absorbe calor. No se debe usar en ambientes cerrados ya que puede provocar intoxicación. No se debe usar directamente sobre una persona ya que puede provocar quemaduras. Útil para incendios tipo A, B y C. Particularmente útil donde hay equipos de alto valor, ya que no los daña.
  \item  Extintor con polvo químico: está compuesto por un agente químico especial para sofocar fuegos. El efecto químico que se produce con las llamas al poner en funcionamiento este tipo de extintores rompe la reacción en cadena del fuego. Además, el fosfato monoamónico que los compone se funde con las llamas y crea una sustancia pegajosa que se adhiere en cualquier elemento sólido, creando una barrera protectora frente a las llamas. Útil contra incendios tipo A, B y C. Pueden originar graves daños en máquinas o equipos delicados (electrónicos). En un lugar cerrado puede dificultar la visión e irritar las vías respiratorias. 
  \item 
   Extintor con espuma química: Al igual que ocurre con los extintores a base de agua, los de espuma ahogan las llamas por enfriamiento y sofocación. En este caso se debe a que la espuma crea una capa continua acuosa que desplaza el aire, enfría e impide posibles escapes de vapor que podrían generar más llamas. Óptimo para fuegos tipo A y B. No se debe usar sobre equipos eléctricos ya que la espuma es conductora. También se produce destrucción de la capa de espuma por combustibles con radicales –OH como el alcohol y no extingue fuegos de derrames.
   \item Ver tabla 1.
    % tablita a
   \begin{table}
\begin{tabular}{|c|c|c|c|c|c|c|}
\hline
 & TIPO A & TIPO B & TIPO C & TIPO D & \begin{tabular}[c]{@{}c@{}}ANTORCHA \\ HUMANA\end{tabular} & \begin{tabular}[c]{@{}c@{}}DAÑOS\\  MATERIALES\end{tabular} \\ \hline
AGUA & SI & NO & NO & NO & SI & SI \\ \hline
CO2 & NO & SI & SI & NO & NO & NO \\ \hline
PQS & SI & SI & SI & SI & NO & SI \\ \hline
ESPUMA & SI & SI & NO & NO & NO & SI \\ \hline
\end{tabular}
\caption{Tabla con información sobre los tipos de extintores.Se muestra en que tipos de fuegos funciona, también indica si se puede apagar una antorcha humana y si el extintor produce daños materiales. }
\end{table}
   
   \item Se debe evitar que la persona corra. La persona debe llevar sus manos a la cara y rodar por el suelo. También se puede ayudar a apagar el fuego utilizando agua, arena, tierra o una prenda grande, de ser posible húmeda, cuidando que no sea acrílica o de fácil combustión. Se debe evitar a toda costa el uso de extintores, solo aplicar en casos de extrema emergencia y de manera muy acotada.
   \item Hay un extintor que se encuentra en dirección hacia la cancha en el laboratorio de superficies, pared mano derecha.
\end{enumerate}

\subsection{Acciones}
\begin{enumerate}[resume]
     \item  Se debe intentar extinguir el fuego solo si es pequeño y controlable, en caso contrario se debe ir al lugar más alejado del fuego, cerrando puertas y en lo posible tapando las rendijas para que no haya circulación de oxígeno ni de humos. En caso de humos se debe tapar boca y nariz con una prenda, húmeda de ser posible y permanecer lo más cercano al piso que sea posible, una vez estando seguro se debe intentar dar aviso a equipos de emergencia.

     \item Se debe ubicar la vía de evacuación más cercana, siempre se debe evacuar por escaleras jamás por ascensor. Si hay humos se debe improvisar una mascarilla y gatear manteniéndose cercano al piso. Hay que tener cuidado al abrir puertas viendo si sale humo por las rendijas y tocándola con la mano, si está caliente significa que hay fuego al otro lado de la puerta, si está fría se debe abrir lentamente asegurándose que no haya humo ni calor. una vez fuera del incendio se debe dar aviso  a equipos de emergencia y no se debe regresar por ningún motivo.
\end{enumerate}
