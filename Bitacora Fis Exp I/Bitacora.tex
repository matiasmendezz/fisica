% Template:     Informe LaTeX
% Documento:    Archivo principal
% Versión:      7.1.7 (31/03/2021)
% Codificación: UTF-8
%
% Autor: Pablo Pizarro R.
%        Facultad de Ciencias Físicas y Matemáticas
%        Universidad de Chile
%        pablo@ppizarror.com
%
% Manual template: [https://latex.ppizarror.com/informe]
% Licencia MIT:    [https://opensource.org/licenses/MIT]

% CREACIÓN DEL DOCUMENTO
\documentclass[letterpaper,oneside]{article}
\usepackage{enumitem} % paquete para hacer q la enumeracion no se corte
\usepackage[table,xcdraw]{xcolor}
\usepackage{amsmath}
\usepackage{tikz}
\usetikzlibrary{circuits.ee.IEC}

\newcommand\Ground{%
\mathbin{\text{\begin{tikzpicture}[circuit ee IEC,yscale=0.6,xscale=0.5]
\draw (0,2ex) to (0,0) node[ground,rotate=-90,xshift=.65ex] {};
\end{tikzpicture}}}%
}
% INFORMACIÓN DEL DOCUMENTO
\def\titulodelinforme {Bitácora}
\def\temaatratar {}

\def\autordeldocumento {Nombre del autor}
\def\nombredelcurso {Física Experimental I}
\def\codigodelcurso {FI3003-2}

\def\nombreuniversidad {Universidad de Chile}
\def\nombrefacultad {Facultad de Ciencias Físicas y Matemáticas}
\def\departamentouniversidad {Departamento de Física}
\def\imagendepartamento {departamentos/dfi}
\def\imagendepartamentoparams {height=1.57cm}
\def\localizacionuniversidad {Santiago, Chile}

% INTEGRANTES, PROFESORES Y FECHAS
\def\tablaintegrantes {
\begin{tabular}{ll}
	Integrantes:
	& \begin{tabular}[t]{l}
		Matías Méndez Zenteno \\
		Gustavo Villar
	\end{tabular} \\
	Profesores:
	& \begin{tabular}[t]{l}
		Claudio Falcón \\
		Victor Fuenzalida
	\end{tabular} \\
	Auxiliares:
	& \begin{tabular}[t]{l}
	    Consuelo Contreras \\
		Gregorio Gonzalez 
	\end{tabular} \\
	& \begin{tabular}[t]{l}
	\end{tabular} \\
	\multicolumn{2}{l}{} \\
	& \\
	\multicolumn{2}{l}{Fecha de realización: \today} \\
	\multicolumn{2}{l}{Fecha de entrega: \today} \\
	\multicolumn{2}{l}{\localizacionuniversidad}
\end{tabular}}{
}

% IMPORTACIÓN DEL TEMPLATE
\input{template}

% INICIO DE PÁGINAS
\begin{document}
	
% PORTADA
\templatePortrait

% CONFIGURACIÓN DE PÁGINA Y ENCABEZADOS
\templatePagecfg

% RESUMEN O ABSTRACT
% \begin{resumen}
% 	\lipsum[1] % Párrafo ejemplo, se puede borrar
% \end{resumen}

% TABLA DE CONTENIDOS - ÍNDICE
%\templateIndex

% CONFIGURACIONES FINALES
\templateFinalcfg

% ======================= INICIO DEL DOCUMENTO =======================

% \begin{titlepage}
   \begin{center}
       \vspace*{9cm}
       \LARGE
       \textbf{Seguridad 2: Fuego}

      
   
            
       \vspace{0.8cm}
     
      
   \end{center}
\end{titlepage}
\section{Fuego }
\subsection{Conceptos}
\begin{enumerate}
  \item Según la Academia Nacional de Bomberos de Chile(ANB), el fuego es una reacción química continua, con generación de luz y calor, en la que se combinan elementos combustibles (agentes reductores) y el oxígeno (agente oxidante). Para esto se requiere la presencia de una fuente de calor y cantidades adecuadas de combustibles y oxígeno. 
  \item %Pregunta 2
  \begin{enumerate}
      \item Fuego y llama no son sinónimos, técnicamente la llama es la parte visible del fuego, es decir son los gases incandescentes que se desprenden de la combustión. Esta se da cuando la temperatura generada es suficientemente alta como para ionizar los vapores a su alrededor. 
      \item Puede haber fuego sin llama, este se llama fuego incandescente. Esto ocurre cuando el combustible, el oxígeno y el calor están en cantidades apropiadas. El oxígeno que participa en la combustión es relativamente pequeño, y está en contacto con la superficie del combustible, el cual debe estar en estado sólido.  Ejemplo: la brasa, que corresponde al resultado de los momentos finales de combustión de la madera. Si soplamos la brasa aumenta la cantidad de oxígeno, con lo que se puede producir llamas.  
  \end{enumerate}
  
  \item  %Pregunta 3
  \begin{enumerate}
      \item \textbf{Triángulo de fuego}: el fuego incandescente necesita tres elementos indispensables para su combustión. Cada lado del triángulo simboliza uno de estos elementos: \textbf{combustible}, \textbf{comburente} y \textbf{calor}.
      \\El \textbf{combustible} es cualquier material que pueda actuar como agente reductor(que cede electrones ), es decir toda sustancia susceptible a arder. 
      \\El \textbf{comburente }es el agente oxidante(que recibe electrones), es cualquier sustancia que en ciertas condiciones de temperatura y presión puede combinarse con un combustible, provocando así una combustión. Actúa oxidando el combustible, y por lo tanto siendo reducido por este último. En general se usa el oxígeno como comburente. 
      \\El \textbf{calor} es una forma de energía presente en todos los objetos materiales, aunque en mayor o menor cantidad. La cantidad de calor se expresa mediante la temperatura. El calor siempre se transfiere de los cuerpos con mayor temperatura a los que están a una temperatura menor. 
      \\Al eliminar uno de los componentes del triángulo, el fuego incandescente se extingue. En cambio, si se aumenta la cantidad de oxígeno, se producen llamas con lo que ya no se puede explicar con el triángulo de fuego, así que se usa el tetraedro de fuego. 
      \item \textbf{Tetraedro de fuego}: es una pirámide triangular en que cada una de sus cuatros superficies se identifica cada unos de los componentes para que el fuego continúe y se produzcan llamas, además de los tres componentes anteriores (triángulo de fuego) es necesaria una reacción en cadena. Esta última es la que explica la producción y mantenimiento de las llamas.  
    \\Para que la combustión con llama se sostenga, el fuego original tiene que generar suficiente calor como para garantizar la existencia de vapores. Los nuevos vapores, al mezclarse con el oxígeno, generan una llama mayor, con más calor, lo que a su vez generan más vapores, y así sucesivamente. En este proceso, una parte del calor producido se transmite al medio ambiente, pero otra vuelve al mismo proceso (lo “retro-alimenta”), produciendo la Reacción en Cadena. Puede ocurrir cuando los otros tres elementos se encuentran en la proporción adecuada.	
  \end{enumerate}
  
  \item Tanto el triángulo de fuego como el tetraedro nos dicen que el fuego se puede generar o mantener al estar todos estos componentes unidos, por lo que al quitar uno de los componentes se puede detener o evitar el fuego.
  
  \item Según norma NCH 934, los tipos de fuego son: %pregunta 5
  \begin{enumerate}
      \item Clase A: Fuegos de materiales combustibles sólidos comunes: madera, papel, telas, diversos plásticos, etc. En general son todos aquellos materiales que al arder dejan brasas o cenizas. Se extinguen mediante el enfriamiento, es decir, eliminando la componente temperatura. Para esto, el agua es la sustancia extintora ideal.
      \item Clase B: Fuego de líquidos combustibles o inflamables: aceites, alcohol, grasas, alquitrán, pinturas, gases inflamables. Se apagan eliminando el oxígeno o interrumpiendo la reacción en cadena.
      \item Clase C: Fuego en equipos eléctricos o materiales energizados. Involucran equipos eléctricos o cualquier otro combustible (fuego clase A,B o D) energizado. El agente extintor no debe ser conductor de electricidad como el agua o la espuma que contiene agua. Una vez desenergizado y verificada la ausencia de electricidad, se puede extinguir con agua. Si no es posible desenergizar, sólo usar agentes extintores no conductores de la electricidad como el Polvo Químico Seco o el Dióxido de Carbono.
      \item Clase D: Fuego en metales combustibles:  magnesio, titanio, litio, zirconio,  sodio, potasio, etc; que al arder alcanzan temperaturas muy elevadas(2700°C a 3300°C).  Se apagan con un matafuego cargado con agente extintor de polvo.
  \end{enumerate}
  \item % pregunta 6
  \begin{enumerate}
      \item La Temperatura de Gasificación es la temperatura mínima a la cual un combustible sólido o líquido desprende vapores en cantidad suficiente para formar una mezcla inflamable con el aire ambiente y permitir la combustión. Los gases no necesitan temperatura de gasificación porque ya están en dicho estado.
      \item Si la temperatura del gas aumenta, por estar en contacto con un material que tenga una determinada temperatura, llegará a un punto en que comenzará a arder con una combustión sostenida.  La temperatura de ignición es la temperatura mínima a la cual los vapores del combustible comienzan a arder.
  \end{enumerate}
  \item %pregunta 7
  \begin{enumerate}
      \item 
      \begin{itemize}
      \item Líquido inflamable: son aquellos cuya Temperatura de Gasificación es inferior a 37 grados Cº. Por ejemplo: la gasolina y el alcohol etílico.
      \item Líquido combustible: son aquellos cuya Temperatura de Gasificación es igual o superior a 37 grados Cº. Por ejemplo: el queroseno y el petróleo.
      \end{itemize}
      \item Ejemplos de combustibles no inflamables son la madera o  parafina a temperatura ambiente.
  \end{enumerate}
  \item Es el rango en cual el porcentaje del vapor del combustible en mezcla con el aire permite la combustión. Los porcentajes maximo y minimos de vapor combustible que permiten la combustión se denominan límites superior e inferior respectivamente. fuera de estos límites no hay combustión.
  \item %pregunta 9
  \begin{enumerate}
      \item La explosión volumétrica es causada por la ebullición violenta de agua, el vapor de agua ocupa un volumen 1800 veces mayor que el agua por lo que la burbuja de vapor desplaza la materia en combustión, produciendo así una explosión. Esto ocurre, cuando se trata de apagar con agua un incendio que no se debe apagar de ese modo.
      \item Un ejemplo doméstico es al verter agua en aceite hirviendo.
  \end{enumerate}
  \end{enumerate}
\subsection{Peligros}

\begin{enumerate}[resume]
  \item  Si los materiales están compuestos por celulosas o fibras artificiales, el humo será de color gris y negro en los plásticos , petróleo, materiales acrílicos. Y puede indicar que el fuego arde en presencia de poco oxígeno. La oscuridad se puede producir por la generación de humos y también por un eventual corte de energía eléctrica.\\ \\
  La oscuridad produce desorientación, lo que dificulta una evacuación rápida y segura. De igual forma dificulta el trabajo de los equipos de extinción y la extracción de los posibles heridos en el incendio. 
  
  \item El humo es el resultado de la combustión incompleta. Las principales componentes del humo son: vapor de agua, hidrocarburos, dióxido de carbono, monóxido de carbono, hollín y otros gases (cianuro de hidrógeno, cloruro de hidrógeno, fosgeno, fosfina, dióxido de azufre, amoníaco, cloro, etc ) que dependen de la composición del material que se quema. \\ \\
  Los peligros que provocan el humo son la asfixia, pérdida de conciencia, dificultad de la visión lo que también provocaría desorientación, en general estos gases están a altas temperaturas por lo que pueden provocar quemaduras.
  \item Lo ideal y crucial es una reacción rápida de la persona para la oportuna extinción del incendio. En caso de que la persona sea incapaz de apagar el incendio por sí sola, debe evacuar el lugar inmediatamente de forma rápida y segura, agachado con un paño humedecido tapando nariz y boca. 
  \item 
  El calor es otro de los peligros que afectan a las personas en un incendio.En un incendio las temperaturas pueden alcanzar los miles de grados por lo que el riesgo de quemaduras graves es muy alto. Hay que tener especial cuidado al abrir puertas ya que en general no se sabe con certeza a qué lugares ha llegado el fuego. Si la visión se ve afectada lo mejor es estar lo más cercano al piso ya que los gases a altas temperaturas permanecen en la altura.
\end{enumerate}

\subsection{Extintores}
Los extintores son un aparato portátil para apagar fuegos o incendios de pequeña magnitud que consiste en una especie de botella grande en cuyo interior hay una sustancia líquida, espumosa o en forma de polvo, para apagar el fuego se arroja un chorro de esta sustancia.
\begin{enumerate}[resume]
  
  \item  Extintor en base a agua: consiste en agua mezclada con algún aditivo químico que sofoque el fuego. Su función principal es absorber el calor. Es útil contra incendios clase A.
  \item Extintor con dióxido de carbono: contribuye a disminuir el oxígeno del aire por lo que este extintor elimina el comburente, al salir el gas alcanza temperaturas muy bajas por lo que también absorbe calor. No se debe usar en ambientes cerrados ya que puede provocar intoxicación. No se debe usar directamente sobre una persona ya que puede provocar quemaduras. Útil para incendios tipo A, B y C. Particularmente útil donde hay equipos de alto valor, ya que no los daña.
  \item  Extintor con polvo químico: está compuesto por un agente químico especial para sofocar fuegos. El efecto químico que se produce con las llamas al poner en funcionamiento este tipo de extintores rompe la reacción en cadena del fuego. Además, el fosfato monoamónico que los compone se funde con las llamas y crea una sustancia pegajosa que se adhiere en cualquier elemento sólido, creando una barrera protectora frente a las llamas. Útil contra incendios tipo A, B y C. Pueden originar graves daños en máquinas o equipos delicados (electrónicos). En un lugar cerrado puede dificultar la visión e irritar las vías respiratorias. 
  \item 
   Extintor con espuma química: Al igual que ocurre con los extintores a base de agua, los de espuma ahogan las llamas por enfriamiento y sofocación. En este caso se debe a que la espuma crea una capa continua acuosa que desplaza el aire, enfría e impide posibles escapes de vapor que podrían generar más llamas. Óptimo para fuegos tipo A y B. No se debe usar sobre equipos eléctricos ya que la espuma es conductora. También se produce destrucción de la capa de espuma por combustibles con radicales –OH como el alcohol y no extingue fuegos de derrames.
   \item Ver tabla 1.
    % tablita a
   \begin{table}
\begin{tabular}{|c|c|c|c|c|c|c|}
\hline
 & TIPO A & TIPO B & TIPO C & TIPO D & \begin{tabular}[c]{@{}c@{}}ANTORCHA \\ HUMANA\end{tabular} & \begin{tabular}[c]{@{}c@{}}DAÑOS\\  MATERIALES\end{tabular} \\ \hline
AGUA & SI & NO & NO & NO & SI & SI \\ \hline
CO2 & NO & SI & SI & NO & NO & NO \\ \hline
PQS & SI & SI & SI & SI & NO & SI \\ \hline
ESPUMA & SI & SI & NO & NO & NO & SI \\ \hline
\end{tabular}
\caption{Tabla con información sobre los tipos de extintores.Se muestra en que tipos de fuegos funciona, también indica si se puede apagar una antorcha humana y si el extintor produce daños materiales. }
\end{table}
   
   \item Se debe evitar que la persona corra. La persona debe llevar sus manos a la cara y rodar por el suelo. También se puede ayudar a apagar el fuego utilizando agua, arena, tierra o una prenda grande, de ser posible húmeda, cuidando que no sea acrílica o de fácil combustión. Se debe evitar a toda costa el uso de extintores, solo aplicar en casos de extrema emergencia y de manera muy acotada.
   \item Hay un extintor que se encuentra en dirección hacia la cancha en el laboratorio de superficies, pared mano derecha.
\end{enumerate}

\subsection{Acciones}
\begin{enumerate}[resume]
     \item  Se debe intentar extinguir el fuego solo si es pequeño y controlable, en caso contrario se debe ir al lugar más alejado del fuego, cerrando puertas y en lo posible tapando las rendijas para que no haya circulación de oxígeno ni de humos. En caso de humos se debe tapar boca y nariz con una prenda, húmeda de ser posible y permanecer lo más cercano al piso que sea posible, una vez estando seguro se debe intentar dar aviso a equipos de emergencia.

     \item Se debe ubicar la vía de evacuación más cercana, siempre se debe evacuar por escaleras jamás por ascensor. Si hay humos se debe improvisar una mascarilla y gatear manteniéndose cercano al piso. Hay que tener cuidado al abrir puertas viendo si sale humo por las rendijas y tocándola con la mano, si está caliente significa que hay fuego al otro lado de la puerta, si está fría se debe abrir lentamente asegurándose que no haya humo ni calor. una vez fuera del incendio se debe dar aviso  a equipos de emergencia y no se debe regresar por ningún motivo.
\end{enumerate}
 % Seguridad 2
% \begin{titlepage}
   \begin{center}
       \vspace*{9cm}
       \LARGE
       \textbf{Unidad 2: Técnicas de Vacío}

      
   
            
       \vspace{0.8cm}
     
      
   \end{center}
\end{titlepage}
\section{Teoría Cinética de los gases}

\subsection{Propiedades de equilibrio}
 \begin{figure}
        \centering
        \includegraphics[width=0.75\textwidth]{Imagenes/Unidad/U2/Tabla composición del aire atmosférico.jpg}
        \caption{Tabla composición del aire atmosférico. Pregunta 1}
        \label{fig:my_label}
    \end{figure}

    \begin{figure}
        \centering
        \includegraphics[width=0.75\textwidth]{Imagenes/Unidad/U2/Unidades de presion del libro.jpg}
        \caption{Unidades de presión. Pregunta 2}
        \label{fig:my_label}
    \end{figure}
\begin{enumerate}
    
    \setcounter{enumi}{2}
    \item %Pregunta 3 
    Velocidad promedio de las moléculas de un gas ideal en términos de la temperatura:
    \begin{equation}
        < v > = \sqrt{\frac{8k_B T}{m \pi}}
    \end{equation}
    donde $k_B$ es la constante de Boltzmann en Joules/Kelvin, $m$ es la masa de la partícula en kg, y $T$ la temperatura en Kelvin.
    \item %Pregunta 4
     La masa del nitrógeno molecular es  $ m = 2\cdot 2.3258671 \cdot 10^{-26}$  kg y la temperatura ambiente $T = 25$°C$= 298.15$ K. La constante de Boltzmann es $k_B = 1.38064910 \cdot 10^{-23}$ J/K. Así , la velocidad promedio es $v = 475 \pm 0.1$ m/s. Esta velocidad se relaciona con la velocidad del sonido ya que este último utiliza al aire como un medio de propagación.
     
\end{enumerate}
\subsection{Propiedades de transporte}
\begin{enumerate}[resume]
\item %Pregunta 5
% medida de la interacción entre proyectiles o partículas lanzadas contra un centro dispersor
El concepto de sección eficaz, como su nombre indica, se refiere al área efectiva para la colisión. La sección eficaz de un objetivo esférico es
\begin{equation}
    \sigma = 4 \pi r^2
\end{equation}
Sabiendo que el radio del nitrógeno es $r=1.55\cdot 10^{-10}$m. Su sección eficaz es $\sigma =7.5\cdot10^{-20} m^2$

\item % Pregunta 6
Debido que las moléculas están distribuidas de forma aleatoria y se mueven a distintas velocidades implica que recorren distintos “caminos libres”.  Así, el camino libre medio  $\lambda$ corresponde a la distancia promedio que recorre una partícula entre dos colisiones sucesivas. 


\item % Pregunta 7
El camino libre medio se expresa como
\begin{equation}
    \lambda =\frac{1}{ \sigma n_v} 
\end{equation}
donde $n_v$ es el número de moléculas por volumen. Por ley de los gases ideales $n_v = \frac{P}{K_BT}$. De esta forma, 

\begin{equation}
    \lambda = \frac{K_BT}{\sigma P }
\end{equation}
donde $K_B$ es la constante de Boltzmann, $T$ la temperatura, $P$ la presión y $\sigma$ la sección eficaz. 



\begin{figure}
    \centering
    \includegraphics[width=0.55\textwidth]{Imagenes/Unidad/U2/grafp8.jpg}
    \caption{Camino libre medio para el nitrógeno. Preguntas 8 y 9}
    \label{fig:my_label}
\end{figure}

% \item % Pregunta 8
% \item % Pregunta 9

\setcounter{enumi}{8}
\item % Pregunta 9
El camino libre efectivo se define como el límite que tienen las partículas respecto a su camino libre medio ya que estas después no pueden aumentar más debido al envase donde se encuentran. 

\item % Pregunta 10
La conductividad térmica de un gas diatómico esta dada por 
\begin{equation}
    K = 3.4817\cdot 10^{-7} n\lambda \sqrt{mT}
\end{equation}
Para el nitrógeno. En el caso en que el camino libre medio es menor que las dimensiones del recipiente se puede expresar como 
\begin{equation}
    K = 3\cdot 10^{-7} \frac{1}{\sigma}\cdot \sqrt{mT} = 2 \cdot 10^{-3} \sqrt{T}
\end{equation}
En el caso que el camino libre medio es mayor que las dimensiones del recipiente se debe considerar el camino libre efectivo,
\begin{equation}
    K = 3\cdot 10^{-7}\frac{P}{K_BT}\lambda_{ef} \sqrt{mT} = 2 \cdot 10^{16}\frac{P\lambda_{ef}}{\sqrt{T}}
\end{equation}
\begin{figure}
    \centering
    \includegraphics[width=0.55\textwidth]{Imagenes/Unidad/U2/grafp10.jpg}
    \caption{Conductividad térmica del nitrógeno a una temperatura de 300 K. Se considera los casos en que el camino libre medio es menor y mayor que el tamaño del recipiente.}
    \label{fig:my_label}
\end{figure}


\end{enumerate}



\subsection{Otras propiedades}
\begin{enumerate}[resume]
    \item % Pregunta 11
    La absorción es cuando un líquido se disuelve en otro líquido o penetra un sólido. La adsorción es la adhesión de átomos, iones o moléculas de un gas, líquido o sólido disuelto a una superficie. La diferencia entre ellos es que la absorción penetra físicamente mientras que la adsorción se queda en la superficie.  
    \item % Pregunta 12
    Se considera un recipiente cúbico de volumen $V=0.01$ $m^3$. El area del recipiente es $A=6a^2$, donde $a$ es la arista del cubo y su valor es $a=\sqrt[3]{0.01}$. De esta forma $A=0.2785$ $m^2$. Por lo tanto el número de moléculas que forman una monocapa en esta superficie esta dado por
    \begin{equation}
        n_{paredes} = \frac{A}{\sigma} = 3.7 \cdot 10^{18} \quad \text{moléculas}
    \end{equation}
    \item % Pregunta 13
    Por ley de los gases ideales
    \begin{equation}
        n=\frac{PV}{K_{B}T}
    \end{equation}
    Luego, si consideramos $T=300$ K y $V=0.01$ $m^3$ tenemos
    \begin{equation}
        n_{volumen}=\frac{0.01\cdot P}{1.380649 \cdot 10^{-23}\cdot 300} = 2.4 \cdot 10^{18} \cdot P
    \end{equation}
    A presion ambiente se tiene $n_{volumen} = 2.4463 \cdot 10^{23}$ moléculas.
    \item % Pregunta 14
    \begin{equation}
        \frac{n_{paredes}}{n_{volumen}}= \frac{3.713 \cdot 10^{18}}{2.4143 \cdot 10^{18} \cdot P}=\frac{1.5379} {P}
    \end{equation}
    \item % Pregunta 15
    A la presión de 1.5379 pascales el efecto de las partículas en las paredes empieza a predominar sobre el efecto de las partículas en el volumen. 
    \begin{figure}
        \centering
        \includegraphics[width=0.55\textwidth]{Imagenes/Unidad/U2/grafp15.jpg}
        \caption{Cociente entre el número de partículas en las paredes y el número de partículas en el volumen en función de la presión.}
        \label{fig:my_label}
    \end{figure}
    
\end{enumerate}


\subsection{Flujo}
\begin{enumerate}[resume]
    \item % Pregunta 16
    \textbf{Flujo turbulento}: también conocido como flujo caótico, corresponde al movimiento de un fluido que se da en forma caótica, en el que las partículas se mueven desordenadamente y las trayectorias de las partículas se encuentran formando remolinos aperiódicos. Ejemplos, flujo detrás de un obstáculo, choque de dos corrientes de agua. Este flujo se produce cuando se comienza a evacuar una cámara de vacío.
    
    \item % Pregunta 17
    \textbf{Flujo viscoso}: en este flujo existen fuerzas de rozamiento entre las moléculas, lo que provoca resistencia a su movimiento. En un tubo las moléculas del fluido se mueven mas rápido cerca del eje longitudinal y mas lento en las paredes. Ejemplo, la miel.
    \\ \\
    \textbf{Flujo laminar}: Se llama flujo laminar o corriente laminar al movimiento de un fluido cuando éste es ordenado, estratificado o suave. En un flujo laminar, el fluido se mueve en láminas paralelas sin entremezclar y cada partícula de fluido sigue una trayectoria suave, llamada línea de corriente. El flujo laminar es típico de fluidos a velocidades bajas o viscosidades altas. Ejemplo, flujo de agua por un tubo a baja velocidad.
    Este flujo se produce cuando se hace vacío grueso y medio.

    \item % Pregunta 18
     % Falta añadir las aplicaciones de los diferentes tipos de flujo a los sistema de vacio. 
    \textbf{flujo molecular}: flujo en el que hay muy poca colisión entre moléculas y por lo tanto el concepto de viscosidad no tiene sentido. Modo de fluir de un gas a través de un conducto en condiciones tales que el recorrido libre medio supera a la dimensión máxima de la sección recta del conducto. El flujo queda totalmente determinado por las colisiones gas-pared. Este flujo se produce cuando se hace alto y ultra alto vacío.
    
    % Falta añadir las aplicaciones de los diferentes tipos de flujo a los sistema de vacio. 

    \item % Pregunta 19
    %Creo q le falta
    \textbf{Flujo de fuga}:
    es un aumento en la presión de un sistema de vacío. Este aumento de presión puede tener dos fuentes:
    \begin{itemize}
        \item una fuga real, dada por defectos del material y/o áreas de conexión, causadas por poros o grietas por estrés mecánico o térmico. También hay moléculas que penetran por permeación desde el exterior.
        
        \item Fuga virtual, no entran moléculas desde el exterior, se produce gas liberado desde las paredes del propio material del recipiente. Es la liberación de un gas que se disolvió, atrapó, congeló o absorbió en el material.
        
    \end{itemize}
\end{enumerate}

\subsection{Aplicaciones}
\begin{enumerate}[resume]
    \item % Pregunta 20  
    \textbf{Física de altas energías}: permite que una partícula alcance altas velocidades ya que disminuye la colisiones con otras moléculas, átomos o partículas.
    \item % Pregunta 21
    \textbf{Bajas temperaturas}:  al haber muy pocas partículas se evita la transmisión de calor por conducción lo que permite desarrollar sistemas aislados térmicamente.
    \\ 
    \textbf{Vaso Dewar}: su estructura principal consta de una doble pared de vidrio, pintada de plateado, y en el espacio intermedio se produce vacío, cuya función principal es evitar la transferencia de energía por convección y conducción, mientras que el plateado permite reflejar la radiación, ya que la plata es un muy buen reflector y tiene baja emisividad. Últimamente se utilizan también fibras de vidrio en el interior para dicho fin.
    \item % Pregunta 22
    \begin{enumerate}
        \item % Parte a
        \textbf{Evaporación de materiales}:  dada la baja presión la temperatura de evaporación del material disminuye. El vacío produce un aumento del camino libre medio de las moléculas del material lo que permite que estas lleguen al substrato. También evita que el aire contamine la superficie del substrato de esta manera se forma una monocapa sin impurezas. 
        \item % Parte b
        Un evaporador térmico que usa un calentador de resistencia eléctrica para derretir el material y elevar su presión de vapor a un rango útil. \textbf{Esto se hace en un alto vacío}, tanto para permitir que el vapor alcance el sustrato sin reaccionar o dispersarse contra otros átomos en fase gaseosa en la cámara, como para reducir la incorporación de impurezas del gas residual en la cámara de vacío. Obviamente, solo los materiales con una presión de vapor mucho más alta que el elemento calefactor pueden depositarse sin contaminar la película.
        \\ 
        Alto vacío: $\mathrm{10^{-3}}-\mathrm{10^{-7}}$ mbar.
        \item % Parte c
        La presión de vapor no depende de la presión ambiente, solo depende de la temperatura y la naturaleza del material que se esta evaporando (aumenta con la temperatura y en general disminuye con el peso molecular).
 
        \item % Parte d
        Al ser baja la presión del recipiente al vapor le tomara mas tiempo alcanzar la presión de saturación por lo que la tasa de condensación crecerá mas lento, es decir habrá mas evaporación. 
        
    \end{enumerate}
    
    \item % Pregunta 23
    \textbf{Ciencia de superficies}: el vacío permite mantener superficies limpias por largos periodos de tiempo lo cual resulta útil en el estudio de la fricción, la adhesión y la corrosión de las superficies.
    \item % Pregunta 24
    Ampolletas, sellado al vacío de alimentos para su conservación. En la fabricación de semiconductores, se requieren ambientes cuidadosamente controlados al vacío.
\end{enumerate}

\subsection{Prevacío}
\begin{enumerate}[resume]
    \item % Pregunta 25
    Consiste en dos paletas cuyos extremos están en permanente contacto con la superficie interna (estator) y están  unidas a un rotor cuyo eje es excéntrico. Esto está diseñado de forma que el volumen de la cámara cambia a medida que el rotor gira. Cuando el aire entra por el puerto de entrada la cámara se agranda, una vez alcanzado el volumen máximo  se inicia la expulsión del aire, al ir disminuyendo el volumen el aire se comprime lo que produce presión en el puerto de salida.
    \begin{figure}
        \centering
        \includegraphics[width=0.8\textwidth]{Imagenes/Unidad/U2/pregunta 25.jpg}
        \caption{Bomba rotatoria de paletas.  }
        \label{pregunta 25}
    \end{figure}
     Animación bomba rotatoria de paletas: \url{https://www.youtube.com/watch?v=gHDVroOnjOI}
    \item % Pregunta 26
    La bomba de dos etapas consiste en dos bombas en serie,es decir la salida de la primera conecta con la entrada de la segunda. La bomba de dos etapas consigue presiones más bajas, y bombea más rápido que la de una etapa.
    \item % Pregunta 27
    El intervalo de presiones en el que funciona la bomba de una etapa es de $\mathrm{10^5}$ a $\mathrm{10^{2}}$ Pa. Mientras que la de dos etapas alcanza una presion del orden de 1 Pa. 
    \item % Pregunta 28
     La bomba scroll consta de dos superficies en forma de espiral, una esta fija y la otra orbita excéntricamente sin girar, atrapando y bombeando o comprimiendo bolsas de fluido entre los rollos. Puede alcanzar una presión del orden de 1 Pa.
     \begin{figure}
         \centering
         \includegraphics[width=0.4\textwidth]{Imagenes/Unidad/U2/pregunta 28.jpg}
         \caption{Bomba scroll.}
         \label{fig:my_label}
     \end{figure}
     Animación bomba scroll: \url{https://www.youtube.com/watch?v=msd432WvlkI}
\end{enumerate}


\subsection{Alto Vacío y Bombas difusoras}
\begin{enumerate}
\setcounter{enumi}{29}
    \item % Pregunta 30
    En la bomba de difusión se calienta aceite, el vapor del aceite sube por la chimenea y es expulsado en dirección descendente. De esta forma las moléculas de gas de la cámara son llevadas hacia abajo por la transferencia de momentum cuando colisionan con las moléculas de aceite. El aceite choca con las paredes que están refrigeradas y se condensa volviendo a la caldera, mientras las moléculas son succionadas por una bomba de respaldo. Estas bombas pueden alcanzar presiones de $\mathrm{10^{-9}} \sim \mathrm{10^{-10}}$ Pa.
    
    \item % Pregunta 31
    La bomba difusora no tiene capacidad de descargar directamente a la atmósfera, por eso debe ser respaldada por una bomba mecánica que mantenga una presión de salida suficiente para evacuar las moléculas de gas. Esta se necesita para que no hayan colisiones de las moléculas de aceita y se produzca un reflujo de aceite hacia la cámara, se requiere un prevacío que alcance una presión en torno a 25-75 Pa. Si la presión no es suficientemente baja el chorro de aceite no logra alcanzar la pared del recipiente y se produce reflujo. \\ \\
    Animación: \url{https://www.youtube.com/watch?v=ubno1kNtxIQ}
    
    \item % Pregunta 32
    \begin{enumerate}
        \item % Parte a
        \textbf{Corriente de retorno}: corresponde al flujo de aceite hacia la cámara de vacío debido a colisiones que hacen ascender a las moléculas de aceite, esta se puede producir por 
        \begin{itemize}
            \item evaporación del fluido condensado en las paredes superiores de la bomba.
            \item ebullición prematura del aceite condensado antes de entrar en la caldera.
            \item la sobreexcitación del vapor de aceite en el chorro superior  fugas en la tapa del chorro.
            \item colisión de las moléculas de aceite con las moléculas del gas.
        \end{itemize}
        
        \item % Parte b
        \textbf{Pantalla}: son paneles ubicados entre la cámara y la boca de la bomba que disminuyen la corriente de retorno obstruyendo el flujo de las moléculas de aceite hacia la cámara.

        \item % Parte c
        \textbf{Trampa criogénica}: es un dispositivo que condensa el vapor de aceite. Con él se puede evitar la corriente de retorno, además el aceite condensado baja por las paredes arrastrando consigo las moléculas de gas. Su objetivo es evitar que los vapores contaminen el experimento.
    \end{enumerate}
\end{enumerate}

\subsection{Tipos de Vacío}
\begin{enumerate}
\setcounter{enumi}{33}
    \item % Pregunta 34
    \begin{figure}
        \centering
        \includegraphics[width=0.7\textwidth]{Imagenes/Unidad/U2/p34.png}
        \caption{Tabla-gráfico con tipos de vacío, flujos, bombas y aplicaciones típicas en un rango de presión correspondiente.}
        \label{fig:my_label}
    \end{figure}
  
\end{enumerate}


\subsection{Medición de Presiones}

\begin{enumerate}
    \setcounter{enumi}{34}
    \item % Pregunta 35
    \textbf{Medición directa}: se mide directamente la fuerza de una partícula incidente sobre una superficie. \\ \\
    \textbf{Medición indirecta}: se miden propiedades del gas que dependen de la presión. 
    \item % Pregunta 36
    Un manómetro sencillo, consiste en un tubo doblado en forma de U que contiene un líquido, generalmente mercurio, un extremo está abierto a la atmósfera y el otro está conectado al depósito donde está el fluido al cual se quiere medir su presión. El líquido del tubo se desplaza hasta alcanzar un equilibrio, en el equilibrio se puede determinar la presión en función de la diferencia de las alturas de las columnas.
    \item % Pregunta 37
    Consiste en un alambre que es calentado por una corriente constante y expuesto al medio donde se quiere medir la presión, a medida que la presión incrementa la temperatura del alambre disminuye (al transferir energía cinética a las partículas el alambre se enfría) , esta disminución de temperatura se mide a través de una termocupla.
    \\
    Una \textbf{termocupla} es un sensor para medir la temperatura. Consiste en dos metales diferentes unidos por un extremo. Cuando la unión de los dos metales se calienta o enfría se produce un voltaje que se puede correlacionar con la temperatura.
    \item % Pregunta 38
    Un manómetro de cátodo frío ioniza el gas en el sistema aplicando un voltaje muy alto entre un ánodo en forma de anillo y un cátodo formado por dos placas paralelas a ambos lados del anillo. por la aplicación de un campo magnético se logra que los electrones sigan trayectorias helicoidales, lo que aumenta el choque con las moléculas de gas. la corriente de iones que se forma depende del potencial de ionización y del número de choque con los electrones que a su vez depende de la presión del sistema, por lo que se usa la corriente de ionización como medida de la presión.
    Animación: \url{https://youtu.be/TG9vtKK-LLw}
\end{enumerate}


\subsection{Evaporación de un material}

\begin{enumerate}
    \setcounter{enumi}{39}
    %\item % Pregunta 39
    %\item % Pregunta 40
    \begin{figure}
        \centering
        \includegraphics[width=0.7\textwidth]{Imagenes/Unidad/U2/p40.jpg}
        \caption{Diagrama montaje experimental. Pregunta 39}
        \label{fig:my_label}
    \end{figure}
    \item % Pregunta 40
    \begin{enumerate}
        \item $600$ ºC
        \item $1010$ ºC
    \end{enumerate}
\end{enumerate} % Unidad 2
%\begin{titlepage}
   \begin{center}
       \vspace*{9cm}
       \LARGE
       \textbf{Seguridad 3: Riesgo Eléctrico}
       \vspace{0.8cm}
   \end{center}
\end{titlepage}

\section{Electricidad}
\subsection{Configuración de la red eléctrica}
\begin{enumerate}
    \item % Pregunta 1
La línea monofásica corresponde a un sistema de producción, distribución y consumo de energía que se forma por una sola corriente alterna o una fase. Se compone  de 3 cables: vivo, neutro y de tierra.
\begin{itemize}
    \item \textbf{Vivo:} Denotado por la letra L. Transmite la corriente desde un generador o transformador. Su tensión nominal es de 220 Voltios y su tensión real de 210-223 Voltios. Se identifica por un color azul, negro o rojo. 
    
    \item \textbf{Neutro:} Denotado por la letra N. Corresponde al conductor por donde retorna la corriente. Su tensión nominal es de 0 Voltios con respecto a tierra, y su tensión real de 0-5 Voltios con respecto a tierra. Se identifica por el color blanco.
    
\item \textbf{Tierra de protección:} Denotado por G de ground, o $\Ground$. Corresponde al cable de protección, se usa para para conducir la descarga a tierra de algún artefacto en mal estado. Su tensión nominal es de 0 Voltios. Se identifica por un color verde o verde/amarillo. 
\end{itemize}

 El uso más frecuente de la línea monofásica es la distribución para iluminación, motores eléctricos pequeños y calefacción.
\\ \\
El valor eficaz se usa para comparar una corriente 
\\ \\
La NCh Elec. 4/2003, indica en su artículo 8.0.4.15 que: “Los conductores de una canalización eléctrica se identificarán según el siguiente código de colores:
    \begin{itemize}
        \item \textbf{Conductor de la fase 1:} Azul.
        \item \textbf{Conductor de la fase 2:} Negro.
        \item \textbf{Conductor de la fase 3:} Rojo.
        \item \textbf{Conductor de neutro o tierra de servicio:} Blanco.
        \item \textbf{Conductor de protección:} Verde o Verde/Amarillo.
    \end{itemize}
Fuente: SEC

\item % Pregunta 2
La línea trifásica corresponde a un sistema de producción, distribución y consumo de energía que se forma por tres corrientes alternas monofásicas desfasadas de igual frecuencia y amplitud.
\begin{enumerate}
    \item El voltaje es de 220 Voltios entre fase y tierra. 
    \item El desfase entre las fases es de 120 grados. 
    \item Como el voltaje entre dos fases es 380 Voltios entonces el voltaje rms es $\frac{380}{\sqrt{2}} = 269 \pm 0.3 $ Voltios. 
\end{enumerate}
\newpage

\item % Pregunta 3
Ver figura 1.
\begin{figure}
        \centering
        \includegraphics[width=0.7\textwidth]{Imagenes/Seguridad/S3/monofasico-bifasico-y-trifasico.jpg}
        \caption{}
        \label{fig:my_label}
    \end{figure}
\item % Pregunta 4
El interruptor automático es un dispositivo de protección contra sobrecargas y cortocircuitos que tiene la capacidad de actuar cuando detecta la falla sin dañarse, lo cual permite su restablecimiento una vez que se resolvió el inconveniente. Consta de dos mecanismos, uno térmico y otro magnético conectados en serie. Su función es proteger equipos e instalaciones eléctricas.

El mecanismo térmico consiste en una lámina bimetálica que se deforma debido al calor producido por el paso de la corriente. Cuando la corriente es lo suficientemente intensa, la deformación provoca que el circuito se abra interrumpiendo la circulación de corriente. El calentamiento y la deformación del bimetálico son procesos lentos, por eso este mecanismo es apropiado para responde a la sobrecarga de corriente.

El mecanismo magnético consiste en una bobina enrollada sobre un núcleo de material magnético, constituyendo un electroimán. El paso de la corriente produce un campo magnético que desplaza al núcleo del electroimán. Si la corriente es lo bastante intensa, el núcleo acciona el mecanismo y el interruptor se abre. Esto ocurre sin demoras, por lo que este mecanismo es apto para responder a los cortocircuitos.

\item % Pregunta 5
El interruptor diferencial es un aparato capaz de interrumpir o abrir un circuito eléctrico cuando ocurren fallas de aislación en un equipo o instalación eléctrica. Actúa cuando existe una diferencia entre las corrientes entrantes y salientes del circuito. Su función es proteger a las personas. En general están calibradas a 30 mA y su velocidad de respuesta es menor a los 50 ms.

Consiste en un núcleo magnético toroidal al cual están enrolladas la fase y el neutro en sentido contrario mutuamente y además una bobina de detección. Cuando la corriente de entrada es igual a la de salida los campos magnéticos producidos por las bobinas de fase y neutro se anulan, de forma que no hay circulación de corriente en la bobina de detección. Pero si hay alguna fuga a una tierra física, la corriente de entrada y salida no serán iguales, provocando que el campo magnético en el núcleo metálico no sea nulo y produciendo una corriente en la bobina de detección activando un electroimán que abre el circuito.
\item % Pregunta 6
Ver figura \ref{imagenmultiple}.
\begin{images}[\label{imagenmultiple}]{Interruptor diferencial y automático.}
    \addimage[\label{subimagen}]{Imagenes/Seguridad/S3/interruptor diferencial.jpg}{width=7cm}{Interruptor diferencial.}
    \addimage{Imagenes/Seguridad/S3/interruptor automatico.jpg}{width=8cm}{Interruptor automático.}
    
\end{images}
\end{enumerate}
 
\subsection{Efectos Biológicos}
El factor determinante en un accidente eléctrico no es el voltaje aplicado, sino la corriente.
\begin{enumerate}[resume]
     %\setcounter{enumi}{7}
     \item % Pregunta 7
     Los factores primarios, que vendrían siendo los técnicos, corresponden a los que influyen y determinan los efectos de la corriente eléctrica sobre el cuerpo humano. Estos son:
     \begin{itemize}
         \item La cantidad de corriente que fluye sobre a través del cuerpo.
         \item La trayectoria dentro del cuerpo, desde el punto de entrada al de salida.
         \item El tiempo El tiempo durante el cual el cuerpo humano es parte del circuito.
     \end{itemize}
     \item % Pregunta 8
     Los factores secundarios, que vendrían siendo los humanos.
     \begin{itemize}
         \item Edad.
         \item Sexo.
         \item Tamaño y estatura.
         \item Condición física general.
         \item Frecuencia de la corriente.
         \item Piel mojada o transpirada.
     \end{itemize}
     \item % Pregunta 9
     En 1 $cm^2$ de piel seca la resistencia varia entre 15 $k\Omega$ y 1 $M\Omega$. Este valor depende del área de contacto y de la humedad de la piel, en una piel húmeda la resistencia puede bajar hasta el 1\% del valor en la piel seca.
     La resistencia interna del cuerpo es de alrededor de 200 $\Omega$ en los miembros y 500 $\Omega$ en el tronco.
   
     \item % Pregunta 10
     Ver figura \ref{figura pregunta 10}.
     \begin{figure}[H]
         \centering
         \includegraphics[width=0.6\textwidth]{Imagenes/Seguridad/S3/p10.png}
         \caption{Efectos producidos por el paso de una corriente alterna a una frecuencia de 50/60 Hz. Pregunta 10.}
         \label{figura pregunta 10}
     \end{figure}
     \item % Pregunta 11
     Ver figura \ref{figura pregunta 11}.
     \begin{figure}[H]
         \centering
         \includegraphics[width=0.7\textwidth]{Imagenes/Seguridad/S3/p11.png}
         \caption{En el gráfico se muestra el umbral de atrapamiento.  La linea izquierda corresponde a mujeres y la derecha corresponde a hombres}
         \label{figura pregunta 11}
     \end{figure}
     
     \item % Pregunta 12
     Ver figura \ref{figura pregunta 12}.
     \begin{figure}[H]
         \centering
         \includegraphics[width=\textwidth]{Imagenes/Seguridad/S3/p12.png}
         \caption{Gráfico ilustrador de la peligrosidad de la corriente con frecuencia.}
         \label{figura pregunta 12}
     \end{figure}
     \begin{itemize}
     
         \item Zona 1: Imperceptible. Generalmente no se perciben efectos con corrientes de hasta 0,5 mA, sin importar el tiempo que esta circule por el cuerpo humano.
         \item Zona 2: Perceptible. Normalmente aún no se producen efectos fisiológicos dañinos. En esta zona se perciben contracciones musculares o tetanizaciones leves.
         \item Zona 3: Efectos reversibles. Por lo general, aún no existe peligro de fibrilación ventricular. En esta zona se pueden producir tetanizaciones y dificultades para la respiración pero que, limitadas en el tiempo, no significan un peligro de muerte.
         \item Zona 4: Posibilidad de efectos irreversibles. Es muy posible que se produzca fibrilación ventricular. El peligro de paro cardiorrespiratorio es alto. Las quemaduras provocadas son graves.
         
     \end{itemize}
     \item % Pregunta 13
     Suponiendo una resistencia de $500\ \Omega$ y que la corriente fatal es de $50 \ mA$ por ley de ohm tenemos,
     \begin{equation}
         V = 5 \cdot 10^{-2} \cdot 500 = 25 \ V
     \end{equation}
     \item % Pregunta 14
        
     \begin{images}[\label{imagenmultiple}]{Umbral de atrapamiento y fibrilación en funcion de la frecuencia.}
         \addimage[\label{subimagen}]{Imagenes/Seguridad/S3/p14_umbral_atrap.jpg}{width=7cm}{Umbral atrapamiento.}
         \addimage{Imagenes/Seguridad/S3/p14_umbral_fibril.jpg}{width=8cm}{Umbral fibrilación.}
    
    \end{images}
    
    El umbral de atrapamiento para una corriente de 50 Hz es de 10 mA y el umbral de fibrilación es de 40 mA.
    Con esto se puede obtener los umbrales para distintas frecuencias de la siguiente forma.
    
    \begin{equation}
        \text{Umbral a la frecuencia estudiada} = F_f \cdot \text{Umbral a la frecuencia de 50 Hz}
    \end{equation}
     
     Como $F_f$ es siempre mayor a uno, se tiene que los umbrales a la frecuencia estudiada son mayores a los umbrales para 50 Hz y por lo tanto las corrientes a mayores frecuencias son menos peligrosas. Sin embargo estas presentan un mayor efecto térmico.
     
     \item % Pregunta 15
     Nunca tocar a la persona para separarla de la fuente. 
     
     Intentar cortar la corriente. Si no es posible hay que intentar separar a la persona utilizando algún elemento que no sea conductor. 
     
     Una vez separada de la fuente se debe evitar mover a la persona. 
     Si hay perdida de conocimiento se debe acostar a la víctima de lado para evitar que se atragante en caso de vomito u otra secreción. Si además la víctima no respira se debe dar asistencia respiratoria. En caso que tampoco haya pulso cardíaco se debe realizar reanimación cardiopulmonar.
     se debe dar aviso a emergencias en caso de 
     \begin{itemize}
         \item quemaduras graves
         \item pérdida de conocimiento
         \item confusión
         \item dificultad para respirar
         \item arritmias o paro cardíaco
         \item dolor y contracciones musculares
         \item convulsiones
     \end{itemize}
     
\end{enumerate} %Seguridad 3 
\section{Unidad 3: Medición por 4 contactos}

\subsection{Fuentes de error}
\begin{enumerate}
    \item % Pregunta 1
    \textbf{Efecto triboeléctrico:} Las corrientes triboeléctricas son generadas por cargas creadas entre un conductor y un aislante debido a la fricción. Aquí, los electrones libres se desprenden del conductor y crean un desequilibrio de carga que provoca el flujo de corriente. 
    
    El cable de “bajo ruido” reduce en gran medida este efecto. Por lo general, utiliza un aislante interno de polietileno recubierto con grafito debajo del escudo externo. El grafito proporciona lubricación y un cilindro equipotencial conductor para igualar las cargas y minimizar la carga generada por los efectos de fricción del movimiento del cable. Sin embargo, incluso el cable de bajo ruido crea algo de ruido cuando se somete a vibración y expansión o contracción, por lo que todas las conexiones deben mantenerse cortas, alejadas de los cambios de temperatura (que crearían fuerzas de expansión térmica) y preferiblemente sujetadas con cinta adhesiva o atando el cable a un cable. una superficie que no vibre, como una pared, un banco u otra estructura rígida.

      %pagina 66
    \item % Pregunta 2
    \textbf{Efecto piezoeléctrico:} Las corrientes piezoeléctricas se generan cuando se aplica tensión mecánica a ciertos materiales cristalinos.
    
    Para minimizar la corriente debida a este efecto, es importante eliminar las tensiones mecánicas del aislante y utilizar materiales aislantes con efectos mínimos de carga piezoeléctrica y almacenada.

    
    \item % Pregunta 3 
    \textbf{Efecto electroquímico:} Este efecto se produce por productos químicos iónicos que crean baterías débiles entre dos conductores en una placa de circuito. La resistencia del aislamiento puede reducirse drásticamente por alta humedad o contaminación iónica. Las condiciones de alta humedad ocurren con condensación o absorción de agua; La contaminación iónica puede ser el resultado de aceites corporales, sales, fundente de soldadura o residuos de manufacturación. Aunque el resultado principal de estos contaminantes es la reducción de la resistencia del aislamiento, la combinación de alta humedad y contaminación iónica puede formar una ruta conductora o incluso pueden actuar como una celda electroquímica con alta resistencia en serie.
    
    Para evitar los efectos de la contaminación y la humedad, se debe seleccionar aisladores que resistan la absorción de agua y mantengan la humedad a niveles moderados. Además, se debe asegurar que todos los aisladores se mantengan limpios y libres de contaminación. Si los aisladores se contaminan, se debe aplicar un agente de limpieza como metanol a todos los circuitos de interconexión.


    
    \item % Pregunta 4
    \textbf{Ruido de Johnson:} En una resistencia, la energía térmica produce el movimiento de partículas cargadas. Este movimiento de carga produce ruido (voltaje).
    
    Una forma de reducir el ruido es reduciendo la temperatura de la fuente. Enfriando la muestra desde temperatura ambiente alrededor de 293 K a la del nitrógeno líquido, 77 K, disminuye aproximadamente a la mitad el factor de ruido.
    \item % Pregunta 5
    \textbf{Efecto superficial en aisladores (superficie limpia):} Una superficie esta definida por un pequeño número de capas atómicas que delimitan dos materiales en contacto íntimo. Si la superficie está en vacío (limpia) la capa atómica exterior de la superficie da lugar a bandas electrónicas que pueden ser ocupadas por electrones libres, de este modo, al aplicar una diferencia de potencial, como la capa exterior ofrece menor resistencia, la corriente tenderá a transitar mayormente por ésta, y no por la sección transversal efectiva del material y al efectuar mediciones, no se estaría obteniendo la corriente que circula derechamente por el mismo, sino que la que circula por la capa exterior.
    \item % Pregunta 6
    \textbf{Efecto superficial (gases adsorbidos):} En el caso de superficies que no estén en el vacío, las capas externas tienden a adsorber gases, luego al aplicar una diferencia de potencial, la corriente a través de dichos gases producirá su ionización, lo que hará que al quitar el voltaje, generen una corriente que decae lentamente con el tiempo en el circuito. Para reducir esto, se debe en lo posible aplicar voltajes no elevados, para que así los gases no se ionizen y no produzcan este desbalance de carga.

    
    \item % Pregunta 7
     La corriente de fuga es una corriente de error que fluye (fugas) a través de la resistencia de aislamiento cuando se aplica un voltaje.Generalmente se convierte en un problema cuando la impedancia del dispositivo bajo prueba es comparable a la de los aisladores en el circuito de prueba.Para reducir las corrientes de fuga, utilice aisladores de buena calidad, reduzca el nivel de humedad en el entorno de prueba y utilice protectores. El uso de aisladores de buena calidad al construir el circuito de prueba es una forma de reducir las corrientes de fuga.
     
     Los aisladores no son  perfectos, cada uno tiene distintas características. Como por ejemplo un aislador puede ser bueno en cuanto a minimizar el efecto piezoeléctrico pero malo minimizando el efecto triboeléctrico.
     
     \item % Pregunta 8
    
    \item % Pregunta 9 
    \textbf{Efecto termoeléctrico:}
    \begin{itemize}
        \item Este efecto se origina por que los electrones se ven excitados a niveles energéticos de manera diferente dependiendo del material, provocando una diferencia de potencial en la unión de estos y, consecuentemente, creando una corriente de circuito.
        \item El voltaje generado depende de los materiales y la diferencia de temperatura. De esta forma si se conocen los materiales se puede conocer la diferencia de temperatura. Bajo este principio funciona el medidor termopar, que consta de dos metales distintos unidos en un punto, este dispositivo es capaz de medir la diferencia de temperatura en los dos extremos.
        \item Por este efecto, se pueden generar voltajes debido a las diferencias de temperatura entre diferentes partes de un circuito y cuando conductores hechos de materiales distintos se unen.
        \item Para minimizar este efecto se recomienda construir los circuitos usando el mismo material para todos los conductores, mantener las conexiones libres de óxido y minimizar el gradiente de temperaturas poniendo los conductores cerca unos de otros y utilizando aislantes con alta conductividad térmica.
    \end{itemize}

    
    \item % Pregunta 10
    \textbf{Campo magnético:} Los campos magnéticos generan voltajes de error en dos circunstancias: 1) si el campo cambia con el tiempo y 2) si hay movimiento relativo entre el circuito y el campo.
    
    Para minimizar los voltajes magnéticos inducidos, se puede reducir el área del circuito los cables deben tender juntos y blindados magnéticamente y deben estar atados para minimizar el movimiento. El mu-metal, una aleación especial con alta permeabilidad a bajas densidades de flujo magnético y a bajas frecuencias, es un material de blindaje magnético de uso común.

    
    \item % Pregunta 11
    \textbf{Lazos de tierra:}
    es una corriente no deseada que circula a través de un conductor que une dos puntos, que en teoría tendrían que estar al mismo potencial pero que en realidad no lo están. se forman cuando se conectan varios instrumentos juntos a una sola señal de retorno, como por ejemplo un solo cable a tierra. Una pequeña diferencia de potencial puede existir entre distintos puntos causando así un flujo de corriente.
    
    Para evitar esto suele utilizarse lo que se conoce como tierra en un solo punto consistente en fijar a tierra la malla de sólo uno de los extremos del cable de la señal de medida. También se puede evitar utilizando voltímetros con alta impedancia de modo común.
    
    \item % Pregunta 12
    \textbf{Capacidades parasitarias:}
    \begin{itemize}
        \item Es un dispositivo capaz de almacenar carga eléctrica y energía en forma de campo eléctrico. Consiste en dos superficies conductoras separadas por un medio dieléctrico.
        \item El cuerpo humano si es un condensador, ya que almacena cargas eléctricas.
        \item Es una capacidad que deliberadamente forma parte de un circuito. Es decir que es parte esencial del mismo.
        \item Es una capacidad que surge por las imperfecciones en la elaboración de un circuito. Por ejemplo, dos conductores pueden estar muy cerca y sin querer formar una capacitancia parásita.
        \item La impedancia de un condensador esta dada por la expresión $Z=\frac{1}{\omega C}$
    \end{itemize}
\end{enumerate} % Unidad 3



% \section{Unidad 4: Movimiento Browniano granular}
% \section{Unidad 5: Inestabilidad de Faraday}
% FIN DEL DOCUMENTO
\end{document}
