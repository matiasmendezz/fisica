\documentclass[aspectratio=43]{beamer}
\usepackage[spanish]{babel}
\usepackage{tikz}
\usepackage{amsmath}
\usepackage{ dsfont }
\usetikzlibrary{decorations.pathmorphing,patterns}
\input{chapters/preamble}
\setbeamertemplate{caption}[numbered]
\title[FI4105]{Estados topológicamente protegidos 
 \\ en metamateriales mecánicos} %->->->->-> Check hyperref title <-<-<-<-<-
\subtitle{}
\author[M. Méndez]{Felipe Cárdenas, Matías Méndez y Diego Rodríguez
                  \\ Profesor: Claudio Falcón}
\institute[UCH]{
    Departamento de Física%
    \\%
    Universidad de Chile%
} %You can change the Institution if you are from somewhere else
\date{19 de abril, 2021}
\logo{\includegraphics[width= 0.2\textwidth]{dfi.pdf}}

\begin{document}
    
    \frame{\titlepage}
    
    \begin{frame}{Índice}
        \tableofcontents
    \end{frame}
     
     
     
     \section{Introducción}
     \begin{frame}{¿Qué es un aislante topológico?}
    

     \begin{columns}[c] % The "c" option specifies centered vertical alignment while the "t" option is used for top vertical alignment

        \column{.6\textwidth} % Left column and width
        
    
     \begin{itemize}
     
         \item Materiales con un ordenamiento topológico protegido por una simetría no trivial.
         \item Aislante en su interior pero que tiene estados conductores en su superficie.
         \item Su existencia ha sido motivo de un extenso estudio teórico, numérico y experimental.
         \item Capacidad de conducir eléctricamente portadores de carga que no se ven afectados por deformaciones suaves de la forma del material.
     \end{itemize}
    

    \column{.3\textwidth} % Right column and width
        \begin{figure}
            \centering
            \includegraphics[width=0.7\textwidth]{Aislante topologico ejemplo.jpg}
            \caption{Estructura cristalina del $\mathrm{Bi}_2\mathrm{Te}^3$. Fuente: Zhang, H. } 
            \label{Ejemplo de aislante topologico}
        \end{figure}
        
     \end{columns}
    \end{frame}
     
   
    \section{Motivación}
    \begin{frame}{Motivación}
     \begin{itemize}
         \item Esta idea se ha extendido a materiales donde la propagación no es electrónica si no que fonónica, donde destacan los metamateriales mecánicos como candidatos de estudio.
         \item Deberían entonces presentar estados de borde protegidos topológicamente. 
         \item ¿Cómo debiesen estar construidos estos materiales y bajo que reglas deben construirse?
         \item ¿Qué caracteriza estos estados de borde?
     \end{itemize}   
    \end{frame}
    
    \begin{frame}{¿Qué es un metamaterial?}
    Material diseñado para tener propiedades que no se encuentran de forma natural. Estas propiedades provienen de la estructura diseñada y no de su composición. 
    \begin{figure}
            \centering
            \includegraphics[width=0.5\textwidth]{Metamaterial.jpg}
            \caption{Ejemplo de metamaterial. Fuente: Wikipedia.} 
            \label{matematerial}
        \end{figure}
    \end{frame}
   \section{Objetivos}
   \begin{frame}{Objetivos}
       \begin{itemize}
           \item Se busca estudiar teórica, numérica y experimentalmente la estructura de banda en un arreglo de osciladores en 2 configuraciones diferentes:
           \begin{enumerate}
               \item  Resortes lineales en presencia de términos giróscopicos y/o con resonadores internos. 
               \item Resonadores de Helmholtz para ondas de agua. 
           \end{enumerate}
           \item Se analiza analogía entre un arreglo de osciladores cuántico y clásico.
           \item Se estudia la aparición de modos de borde topológicamente protegidos
 

       \end{itemize}
   \end{frame}  
   
   
   \section{Marco Teórico}
   \begin{frame}{Modelo Su-Schrieffer-Heeger (SSH)}
   \begin{figure}
            \centering
            \includegraphics[width=\textwidth]{SSH model.png}
            \caption{Geometría del Modelo SSH. Fuente: J. K. Asboth, L. Oroszlány, A. Pályi.} 
            \label{Ejemplo de aislante topologico}
        \end{figure}
    \begin{equation}
        \hat{H}= v \sum_{m=1}^N (\mid m,B \rangle \langle m,A \mid + h.c.)+w\sum^{N-1}_{m=1}(\mid m,A \rangle \langle m,B \mid + h.c.)
    \end{equation}

\end{frame}
\begin{frame}{Modelo Su-Schrieffer-Heeger (SSH)}
\begin{equation}
\small
\hat{H} =
\begin{pmatrix}

\ddots  &  &  &   &  & \\ 
&0 & v & 0 & 0  &  & \\ 
&v & 0 & w & 0 &  & \\ 
&0 & w & 0 & v &0  & 0\\ 
&0 & 0 &v  & 0 &w  &0 \\ 
& &  &0  & w & 0 &v \\ 
& &  & 0 &  0& v &0 \\
& &  &  &  &  & & \ddots  
\end{pmatrix}
\Rightarrow 
 \mathcal{H}(k) = 
 \begin{pmatrix}
0 & v+w e^{-ikd}\\ 
v+w e^{ikd} & 0
\end{pmatrix} 
\end{equation}
\begin{equation}
   \hat{H}^2= \varepsilon(k)^2 \hat{\mathbb{I}}_2 \Rightarrow  \varepsilon(k) =  \left | v+w e^{-ikd} \right | = \sqrt{v^2+w^2+2vw\cos (kd)}
\end{equation}
\end{frame}

\begin{frame}{Modelo Su-Schrieffer-Heeger (SSH)}
\begin{figure}
            \centering
            \includegraphics[width=0.8\textwidth]{Grafico relacion de dispersion.png}
            \caption{Relaciones de dispersión del modelo SSH. Fuente: J. K. Asboth, L. Oroszlány, A. Pályi.} 
            \label{Grafico de dispersiones}
        \end{figure}
    
\end{frame}



\section{Metodología}
\begin{frame}{Metodología }
Consideremos un arreglo de resortes en una dimensión.
\begin{tikzpicture}
\centering
\tiny
\draw[dashed] (3,-1) -- (7.3,-1) -- (7.3,1) -- (3,1) -- (3,-1); % Dibujo del rectangulo

%Comandos para colocar textos
\node[text width=2cm] at (5.7,-1.4) 
    {Bloque N};
\node[text width=2cm] at (5.7,1.4) 
    {Largo $L$};
\node[text width=2cm] at (2.8,-0.5) 
    {$m$};
\node[text width=2cm] at (4.8,-0.5) 
    {$m$};
\node[text width=2cm] at (7.1,-0.5) 
    {$m$};
\node[text width=2cm] at (9.2,-0.5) 
    {$m$};
\node[text width=2cm] at (1.7,-0.5) 
    {$k_2$};
\node[text width=2cm] at (3.6,-0.5) 
    {$k_1$};
\node[text width=2cm] at (6,-0.5) 
    {$k_2$};   
\node[text width=2cm] at (8.3,-0.5) 
    {$k_1$};
\node[text width=2cm] at (10.3,-0.5) 
    {$k_2$};  


%Comando para dibujar un circulo
\node[circle,fill=black,inner sep=2.5mm] (a) at (2,0) {};
% Comando para hacer el resorte
\draw[decoration={aspect=0.3, segment length=1mm, amplitude=3mm,coil,
pre=lineto,pre length=2mm,post=lineto,post length=2mm},decorate] (0,0) -- (a); 
%se repite (circulo y resortes)
\node[circle,fill=black,inner sep=2.5mm] (a) at (4,0) {};
\draw[decoration={aspect=0.3, segment length=1mm, amplitude=3mm,coil,
pre=lineto,pre length=2mm,post=lineto,post length=2mm},decorate] (2.3,0) -- (a); 
\node[circle,fill=black,inner sep=2.5mm] (a) at (6.3,0) {};
\draw[decoration={aspect=0.3, segment length=1mm, amplitude=3mm,coil,
pre=lineto,pre length=2mm,post=lineto,post length=2mm},decorate] (4.3,0) -- (a); 
\node[circle,fill=black,inner sep=2.5mm] (a) at (8.3,0) {};
\draw[decoration={aspect=0.3, segment length=1mm, amplitude=3mm,coil,
pre=lineto,pre length=2mm,post=lineto,post length=2mm},decorate] (6.6,0) -- (a); 
\node[circle,fill=gray,inner sep=0mm] (a) at (10.3,0) {};
\draw[decoration={aspect=0.3, segment length=1mm, amplitude=3mm,coil,
pre=lineto,pre length=2mm,post=lineto,post length=2mm},decorate] (8.6,0) -- (a); 


\end{tikzpicture}
Modelo análogo al modelo SSH: 
\begin{equation}
    M\ddot{\vec{x}}+ V \vec{x}= 0 \Leftrightarrow i\hbar\partial_t\psi = \hat{H}\psi
\end{equation}
\end{frame}
\begin{frame}{}
\begin{equation}
\small
 V = 
\begin{pmatrix}
\ddots  &  &  &  &  & \\ 
 & k_1+k_2 & -k_1 &0  &0  & \\ 
 &  -k_1&k_1+k_2  & -k_1 &  0& \\ 
 & 0 & -k_1 & k_1+k_2 & -k_1 & \\ 
 &  0& 0 &-k_1  & k_1+k_2 & \\ 
 &  &  &  &  & \ddots 
\end{pmatrix}
\end{equation}

\begin{equation}
    \Rightarrow v(q) = \begin{pmatrix}
0 & k_1+k_2e^{-iqL}\\ 
k_1+k_2e^{iqL} & 0
\end{pmatrix}
\end{equation}
\begin{equation}
\Rightarrow  w(q) = \sqrt{k_1^2+k_2^2+2k_1k_2cos(qL)}
\end{equation}
\end{frame}

 
    % \input{chapters/a-silly-idea.tex} %You can put the frames directly into the presentation, but using the input command and writing them in separate .tex files might be more organized
    
    % \input{chapters/playing-around.tex}
    
    % \input{chapters/fourier-playground.tex}
 
    
    \section*{Referencias} %You can remove this if you do not want to use it
    
        \nocite{Zhang}
        \nocite{Asbáth}
         \nocite{Batra}
         
         \begin{frame}{Referencias}
            \printbibliography
        \end{frame}

    \section{}
    \begin{frame}{}
        \centering
            \Huge\bfseries
        \textcolor{orange}{Continuará}
    \end{frame}
  
\end{document}

% \begin{columns}[c] % The "c" option specifies centered vertical alignment while the "t" option is used for top vertical alignment

%         \column{.6\textwidth} % Left column and width
%         \column{.3\textwidth} % Right column and width
%     \end{columns}