% Template:     Informe LaTeX
% Documento:    Archivo principal
% Versión:      7.0.3 (02/08/2020)
% Codificación: UTF-8
%
% Autor: Pablo Pizarro R.
%        Facultad de Ciencias Físicas y Matemáticas
%        Universidad de Chile
%        pablo@ppizarror.com
%
% Manual template: [https://latex.ppizarror.com/informe]
% Licencia MIT:    [https://opensource.org/licenses/MIT]

% CREACIÓN DEL DOCUMENTO
\documentclass[letterpaper,oneside]{article}

% INFORMACIÓN DEL DOCUMENTO
\def\titulodelinforme {Tarea 1}
\def\temaatratar {Métodos Numéricos para la Ciencia e Ingeniería}

\def\autordeldocumento {Nombre del autor}
\def\nombredelcurso {Métodos Numéricos para la Ciencia e Ingeniería }
\def\codigodelcurso {FI3104}

\def\nombreuniversidad {Universidad de Chile}
\def\nombrefacultad {Facultad de Ciencias Físicas y Matemáticas}
\def\departamentouniversidad {Departamento de Física}
\def\imagendepartamento {departamentos/dfi}
\def\imagendepartamentoescala {0.2}
\def\localizacionuniversidad {Santiago, Chile}

% INTEGRANTES, PROFESORES Y FECHAS
\def\tablaintegrantes {
\begin{tabular}{ll}
	Integrante:
	& \begin{tabular}[t]{l}
		Matías Méndez Zenteno
	\end{tabular} \\
	RUT:
	& \begin{tabular}[t]{l}
		20.243.742-7
	\end{tabular} \\
	Profesor:
	& \begin{tabular}[t]{l}
		Valentino González C.
	\end{tabular} \\
	Auxiliares:
	& \begin{tabular}[t]{l}
		Vicente Donaire Aguilera\\
		Benjamín Navarrete
	\end{tabular} \\

	
 \\
	& \\
	\multicolumn{2}{l}{} \\
	\multicolumn{2}{l}{Fecha de entrega: 26 de octubre de 2020} \\
	\multicolumn{2}{l}{\localizacionuniversidad}
\end{tabular}}{
}

% IMPORTACIÓN DEL TEMPLATE
\input{template}

% INICIO DE PÁGINAS
\begin{document}
	
% PORTADA
\templatePortrait

% CONFIGURACIÓN DE PÁGINA Y ENCABEZADOS
\templatePagecfg

% RESUMEN O ABSTRACT


% TABLA DE CONTENIDOS - ÍNDICE


% CONFIGURACIONES FINALES
\templateFinalcfg

% ======================= INICIO DEL DOCUMENTO =======================
\section{Introducción}
En el presente informe se muestra el desarrollo hecho para la Tarea 1 del curso \textit{Métodos Numéricos para la Ciencia e Ingeniería}. El objetivo de esta tarea es resolver numéricamente la ecuación:
$$ $$
$$ p(x<a)=0.95 = \int^{a}_{-\infty} pdf(x)dx $$
Para esto se considera que $pdf$ es una distribución $\chi^2$:
$$pdf(x)=\chi^2(x)=\frac{1}{2^{k/2} \Gamma(k/2)}x^{\frac{k}{2}-1 }e^{\frac{-x}{2}} $$
donde $\Gamma$ es la función Gamma:
$$\Gamma(z)=\int^{\infty}_0 x^{z-1}e^{-x}dx$$
En la distribución $\chi^2$, $k$ corresponde a 4.742 para esta tarea.

\section{Desarrollo}

Se debe considerar que la distribución $\chi^2$ está definida en $\mathbb{R}^+$ solamente. Por lo que la ecuación a resolver se reescribe como:
$$p(x<a)=0.95 = \int^{a}_{0} pdf(x)dx $$

Primero se debe encontrar un valor para $\Gamma(k/2)$, por lo que se crea la función \textbf{gamma\_1} que mediante un algoritmo calcula $\Gamma(z)$ para cualquier real positivo. Esta función  recibe x como parámetro, el intervalo donde se integra, y N, que corresponde a la cantidad de segmentos en que se divide el intervalo.
\begin{figure}[h]
\includegraphics[width=0.5\textwidth]{gamma sin c.v..pdf}
\caption{El área pintada debajo de la curva corresponde a $\Gamma(k/2)$. Se trata de calcular esta área a través del método del trapecio, pero primero se debe realizar un cambio de variable para regularizar la curva ya que se tiene un límite que infinito.}
    \label{Funcion gamma sin c.v.}
\centering
\end{figure}
Así, se tiene que regularizar

la curva de la función gamma pues esta tiene un límite infinito, esta no es una curva suave como se observa en la figura \ref{Funcion gamma sin c.v.}, por lo que se usa el cambio de variable $y=e^x$ en la integral de forma que esta queda:

$$\Gamma(z)=\int^1_0 ln\begin{pmatrix}
\frac{1}{y}
\end{pmatrix}^{z-1}dy$$

 Para integrar $\Gamma(z)$ se utiliza el método del trapecio, se  integra en el intervalo (0,1) y se divide este en N segmentos. La gracia de este cambio da variable es que la función a integrar es más suave como se observa en la figura \ref{Funcion gamma con c.v.}.
\begin{figure}[h]
\includegraphics[width=0.4\textwidth]{gamma con c.v..pdf}
\caption{El área pintada bajo la curva corresponde a $\Gamma(k/2)$luego del cambio de variable. Esta curva ya está regularizada, pero ahora se tiene que esta se indefinida en $x=0$, por lo que se trata de arreglar esto aproximando el 0 a $10^{-10}$. Para luego usar el método del trapecio para calcular el área bajo la curva.}
    \label{Funcion gamma con c.v.}
\centering
\end{figure}
 Luego, para mejorar la precisión numérica de la integral usamos la función \textbf{precision\_gamma}, un algoritmo que recibe la función, z, el N con que se parte y la tolerancia, como parámetros. Y usa la convergencia relativa para encontrar la mejor aproximación numérica de la integral. Si denotamos $I_N$ como la integral numérica para N segmentos en el método del trapecio, tenemos que la función busca cumplir:
 $$ \begin{vmatrix}
\frac{I_{2N}-I_N}{I_n}
\end{vmatrix}
<\epsilon$$
 Para esta tarea se busca solamente el resultado numérico para $\epsilon=10^-6$, ya que se considera estudiar más la integral principal, $\int^a_0pdf(x)dx$.
 \\ \\
 Teniendo en cuenta la precisión numérica de $\Gamma(k/2)$, se procede a calcular la función 
 $$f(a)=\int^a_{0}\frac{1}{2^{k/2} \Gamma(k/2)}x^{\frac{k}{2}-1 }e^{\frac{-x}{2}}dx $$
 Para integrar numéricamente esta función se procede igual que antes, se usa el método del trapecio en la función \textbf{f}, y para mejorar su precisión numérica se utiliza la función \textbf{precision\_f}. Estas funciones funcionan de manera análoga a las anteriores.
 
 Así, para resolver $f(a)=0.95$ se define
 $$g(x)=f(x)-0.95$$
 De esta forma, la raíz de g(x) es la solución para la ecuación. Y así. podemos ocupar el método de la \textbf{bisección} en g(x) para encontrar la raíz numéricamente. Para ello se necesita como parámetro un intervalo que encierre la solución, y una tolerancia para la precisión de la solución numérica; así se busca en el gráfico de g(x) de la figura \ref{grafico g(x)} un intervalo.
 
 \begin{figure}[h]
\includegraphics[width=0.5\textwidth]{grafico g(x).pdf}
\caption{Del gráfico de g(x) se observa que la raíz esta encerrada entre 7.5 y 12.5, por lo que para el método de la bisección podemos usar el intervalo (7.5, 12.5) para comenzar a buscar la raíz.}
    \label{grafico g(x)}
\centering
\end{figure}
 Se usa el método de la bisección por su sencillez para implementarla, y por asegurar la convergencia al tener que los extremos evaluados del intervalo tienen signos distintos. 
 
 
 \section{Resultados}
 Para comparar los resultados se calculó también $\Gamma(k/2)$ y $a$ usando la librería scipy, implementándose así los métodos del trapecio y bisección, junto a las funciones gamma y $\chi^2$.\\
 A continuación se ven algunas tablas para los valores calculados.
 \begin{table}
 \caption{Se muestran comparaciones para distintos valores evaluados en la función gamma creada y la función gamma de scipy.}
\begin{tabular}{|c|c|c|}
\hline
\textbf{z} & \textbf{$\Gamma(z)$ calculado} & \textbf{$\Gamma(z)$ scipy} \\ \hline
1          & 0.9999999999000001          & 1.0                     \\ \hline
2          & 1.0000002510990291          & 1.0                     \\ \hline
k/2=2.371  & 1.2191370224729854          & 1.2191365664602392      \\ \hline
3          & 2.0000007754838287          & 2.0                     \\ \hline
4          & 6.000000848546728           & 6.0                     \\ \hline
\end{tabular}
\end{table}

 \begin{table}
\begin{tabular}{|c|c|c|}
\hline
\textbf{$a$ calculado}          & \textbf{Convergencia} & \textbf{Tiempo de ejecución {[}ms{]}} \\ \hline
10.668931007385254 & $10^{-5}$          & 7383                                  \\ \hline
10.668907165527344 & $10^{-4}$          & 7087                                  \\ \hline
10.66864013671875  & $10^{-3}$          & 6663                                  \\ \hline
10.6689453125      & $10^{-2}$          & 6497                                  \\ \hline
\end{tabular}
\caption{Se muestra la estimación de $a$ para distintas convergencia junto a su tiempo de ejecución}
\end{table}

\begin{table}
\begin{tabular}{|l|l|}
\hline
\textbf{$a$ scipy} & \textbf{Precisión}            \\ \hline
10.66889796798364  & $10^{-12}$ \\ \hline
10.668897967981366 & $10^{-11}$                    \\ \hline
10.668897967908606 & $10^{-10}$                    \\ \hline
10.668897965982751 & $10^{-9}$                     \\ \hline
\end{tabular}
\caption{Se muestra la estimación de $a$ para distintas ´convergencias usando los algoritmos de la librería scipy.}
\end{table}
Finalmente se muestra un gráfico de $g(x)$ junto a las raíces numéricas calculadas.
\begin{figure}
    \centering    
    \includegraphics[width=0.6\textwidth]{grafico g(x) con raices numericas con zoom.pdf}
    \caption{Para la función $g(x)$, se observan las raíces numéricas calculadas para distintas convergencias relativas usando el método de la bisección.}
    \label{grafico de g con raices y zoom}
\end{figure}{}
\newpage
\section{Discusión y Conclusiones}

Respecto a la función Gamma calculada, el cambio de variable realizado no logra resolver el límite infinito en total, pero logra suavizar la curva. En la tabla 1 se observa que la función tiene una precisión de 5 decimales comparada con la función Gamma de scipy, por lo que es bastante buena esta aproximación para lo que buscamos. \\ \\
Sobre la solución encontrada, se tiene que la precisión de esta va relacionada al método del trapecio aplicado a $f(x)$, y al método de la bisección usado para encontrar la solución sobre ella. Así, junto a la tabla 2 y la figura 4, se observa que la solución numérica comienza a fallar respecto a la precisión luego de los 3 decimales. Ya que en el gráfico se observa que la solución $a$ con una convergencia de $10^{-2}$ esta más cerca de la solución real que $a$ con una convergencia de $10^{5}$. Esto se puede deber a las tolerancias usadas para los métodos implementados.
\\ \\
Las soluciones numéricas de $a$ utilizando los métodos de scipy muestra que coinciden en los primeros tres decimales con los $a$ calculados.
\\ \\
Sobre los tiempos de ejecuciones para las distintas precisiones de $a$ se tuvo que el algoritmo no presentó una gran diferencia de tiempo para distintas convergencias. \\ \\
Finalmente, con todo esto se logró implementar una solución para la ecuación, que por lo analizado obtuvo una precisión de 3 decimales.

  %he trapezoidal rule has faster convergence in general than Simpson's rule
%justificar porque se utiliza el metodo del trapecio, cual es su ventaja
%Seguir luego con explicar el uso de la funcion precision_gamma. Esta da una convergencia relativa que servira despues para los resultados.
%Graficar la funcion gamma con y sin el cambio de variable. O tal vez no
%Luego sigue explicar la integracion de chi, es igual que el de gamma
%y luego explicar el metodo de la biseccion y sus ventajas.
%MOstrar grafico donde se encuentra el cero
%Ver el tema de las iteraciones y el tiempo
%Comparar con los metodos de las librerias
%EL CAMBIO DE VARIABLE NO RESULTA TAN SUAVE PERO AL MENOS ES MEJOR QUE %LA OTRA CURVA
% FIN DEL DOCUMENTO
\end{document}
\begin{figure}
    \centering    
    \includegraphics[width=0.6\textwidth]{}
    \caption{}
    \label
\end{figure}{}